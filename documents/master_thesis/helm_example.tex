\chapter{Holomorphic Embedding Load Flow Example}
\label{chap:helm_example}

\begin{figure}
	\centering
	\begin{circuitikz}	
	\draw (0, 0) node[above] {\SI{1}{V}} to [R=\SI{1}{$\Omega$},*-*] (4, 0);
	\draw[-stealth] (4, 0) --++ (0,-1);
	\draw (4.5, -1) node {$P$};
\end{circuitikz} 

	\caption{Power net with two nodes}
	\label{fig:two_node_net}
\end{figure}

For the sake of simplicity only the small net \figref{two_node_net} is calculated, which has because of only real valued input parameter ($U_1 = \si{1}{V}$, $Z = \si{1}{\Omega}$ and $P = \si{0,23}{W}$) also no imaginary parts in the solution and the intermediate results. The exact solution is determined with the current sum on the second node
\begin{equation}
	\frac{U_1 - U_2}{Z} = \frac{P}{U_2}.
\end{equation}
This is only one quadratic equation
\begin{equation}
	U_2^2 - U_1 U_2 + P Z = 0,
\end{equation}
whereas the physical solution is
\begin{align}
	U_2 & = \frac{U_1 + \sqrt{U_1^2 - 4 P Z}}{2} \\
		& = \frac{\si{1}{V} + \sqrt{(\si{1}{V})^2 - 4 \cdot \si{0,23}{W} \cdot \si{1}{\Omega}}}{2} = \si{0,641421356}{V}.
\end{align}

The first 15 coefficients and the tableau of the analytic continuation for this example can be found in \tabref{helm_example_data}.

\begin{table}[h]
	\tiny
	\begin{tabular}{|l|l|l|l|l|l|}
		\hline
		n	& $c_n$					& $\epsilon_0^{(n)}$	& $\epsilon_1^{(n)}$	& $\epsilon_2^{(n)}$	& $\epsilon_3^{(n)}$ \\ \hline
		0	& 1						& 1						& -4,347826087			& 0,7012987013			& -53,0779978631 \\
		1	& -0,23					& 0,77					& -18,9035916824		& 0,672037037			&  -123,1056206917 \\
		2	& -0,0529				& 0,7171				& -41,094764527			& 0,6598435294			& -228,1834058046 \\
		3	& -0,024334				& 0,692766				& -71,4691556991		& 0,6534624888			& -377,2251277719 \\
		4	& -0,01399205			& 0,67877395			& -110,9769498434		& 0,6497065945			& -581,3877511012 \\
		5	& -0,0090108802			& 0,6697630698			& -160,8361591933		& 0,647328765			& -854,33787053 \\
		6	& -0,0062175073			& 0,6635455625			& -222,5006154848		& 0,6457460789			& -1212,6396784013 \\
		7	& -0,0044943696			& 0,6590511929			& -297,6596862673		& 0,6446531589			& -1676,2347927996 \\
		8	& -0,0033595413			& 0,6556916516			& -388,2517646964		& 0,6438767511			& -2269,0112954398 \\
		9	& -0,0025756483			& 0,6531160033			& -496,4856326042		& 0,6433125844			& -3019,4684373807 \\
		10	& -0,002014157			& 0,6511018463			& -624,8675009892		& 0,6428949783			& -3961,4878350678 \\
		11	& -0,0016003393			& 0,6495015071			& -776,2329204835		& 0,6425810317			& -5135,2249109319  \\
		12	& -0,0012882731			& 0,6482132339			& -953,7833616527		& 0,6423418797			& \\
		13	& -0,0010484561			& 0,6471647778			& -1161,1275707076 		&						& \\
		14	& -0,0008612318			& 0,646303546 			&						&						& \\ \hline \hline
		n	& $\epsilon_4^{(n)}$	& $\epsilon_5^{(n)}$	& $\epsilon_6^{(n)}$	& $\epsilon_7^{(n)}$	& $\epsilon_8^{(n)}$ \\ \hline
		0	& 0,6577569578			& -257,6917202828		& 0,6463316434			& -965,4309176106		& 0,6429378349 \\
		1	& 0,65032677			& -507,9966791024		& 0,6441455368			& -1793,4498231255		& 0,6422671252 \\
		2	& 0,6467529581			& -891,5173618297		& 0,6430368063			& -3092,6891389693		& 0,641918518 \\
		3	& 0,6448085383			& -1455,9368172673		& 0,6424258404			& -5063,8222472924		& 0,6417255516 \\
		4	& 0,6436650919			& -2262,8754903897		& 0,6420688182			& -7977,0096092318		& 0,6416135118 \\
		5	& 0,6429551357			& -3391,1393489018		& 0,6418507571			& -12192,0568935985		& 0,641545953 \\
		6	& 0,6424961042			& -4940,6930609089		& 0,641712852			& -18183,7051794566		& 0,6415039389 \\
		7	& 0,6421897747			& -7037,4691049271		& 0,6416231356			& -26573,1993310289 	& \\
		8	& 0,6419800633			& -9839,157482384		& 0,6415633772 			&						& \\
		9	& 0,641833429			& -13542,1506659578 	&						&						& \\
		10	& 0,6417290521 			& 						& 						& 						& \\ \hline \hline
		n	& $\epsilon_9^{(n)}$	& $\epsilon_{10}^{(n)}$	& $\epsilon_{11}^{(n)}$	& $\epsilon_{12}^{(n)}$	& $\epsilon_{13}^{(n)}$ \\ \hline
		0	& -3284,407797319		& 0,6418935502			& -10759,3888854953		& 0,6415687586			& -34732,5003384756 \\
		1	& -5961,2467473499		& 0,6416851362			& -19352,1012790472		& 0,6415037407			& -62254,6325981286 \\
		2	& -10246,0728069144		& 0,6415753189			& -33322,845655593		& 0,6414691767 			& \\
		3	& -16902,4129690484		& 0,6415144192			& -55425,9705199859 	&						& \\
		4	& -26993,9749727727		& 0,6414792475 			&						&						& \\
		5	& -41985,2383684168 	&						&						&						& \\ \hline
	\end{tabular}
	\caption{HELM example coefficients and tableau}
	\label{tab:helm_example_data}
\end{table}

As one can see, the result of \si{0,6414674063}{V} with HELM is already close to the exact solution of \si{0,641421356}{V}, although only 15 coefficients were calculated. This is caused by the very special and minimal example. For realistic problems the analytic continuation is absolutely necessary to get reasonable results.