\chapter{Introduction}
The energy revolution, driven by consumers, politics and the society at whole, is stressing our power supply system. For instance, the decision to shut down the atomic power plants in germany in the foreseeable future will shift the main power sources from the south of germany into the north. Historically these sources were placed close to the main loads in the power net, but with the need to use renewable energy sources our power plants will be located where this energy sources are available, not where the big cities and the industry was built. Therefore, the load in the power net will increase and consequently the need to decide wether a certain change in the power net can be made without a risk of crashing the power supply system.
The tool of choice for this decision is a load flow calculation. Throughout the past centuries there were three main algorithms used for these calculations: Current Iteration \secref{current_iteration}, Newton-Raphson \secref{newton_raphson} and the Fast-decoupled-load-flow (FDLF) \secref{fdlf}. All these algorithms have two things in common: They are iterative and can not guarantee to find a physical correct solution. With this drawback in mind a totally new approach was developed by Antonio Trias in \citep{helmIEEE}, called Holomorphic Embedding Load Flow (HELM), described more detailed in \secref{helm}. This approach guarantees to find a solution for a given load flow problem if, and only if, the system is not collapsing. Unfortunately, this method is implemented, as far as I know, in HELM-Flow \footnote{http://www.gridquant.com/solutions/helm-flow/} by Gridquant. The very common tool PSS SINCAL \footnote{http://www.simtec.cc/sites/sincal.asp} has not yet implemented this new approach to the load flow problem, but it is the tool of choice for instance in the Institute of Power Transmission Systems at the Technische Universität München. Throught this work I will accomplish:
\begin{itemize}
	\item As PSS SINCAL comes with a well documented data format I will implement a tool which parses this format, calculates the node voltages with HELM and writes the results back into this format. \secref{link_sincal}
	\item To test, if HELM is useful in terms of accuracy and runtime, I will then use the available models of the german power grid in \secref{large_scale_powernets}. For these available models it is not yet known if they are stable or if they would collapse, as the iterative algorithms are not able to calculate them. HELM will hopefully give us a little bit more insight into this aspect.
\end{itemize}