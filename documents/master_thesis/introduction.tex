\chapter{Introduction}
The energy revolution, driven by consumers, politics and the society at whole, is stressing our power supply system. For instance, the decision to shut down the atomic power plants in germany in the foreseeable future will shift the main power sources from the south of germany into the north. Historically these sources were placed close to the main loads in the power net, but with the need to use renewable energy sources our power plants will be located where these energy sources are available, and not where the big cities and the industry was built. Therefore, our power nets will have to change to be able to transport the energy from the sources to the loads.

One of the necessary steps for a change in the power net is a static load-flow calculation, for which we had only algorithms with a bad convergence behaviour during the past. These algorithms have one in common: They are iterative and can not guarantee to find the physical correct solution. With this drawback in mind a totally new approach was developed by Antonio Trias in \citep{helmIEEE}, called \emph{Holomorphic Embedding Load Flow} (\emph{HELM}), described more detailed in \secref{helm}. This approach guarantees to find a solution for a given load-flow problem if, and only if, the system is stable. Unfortunately, this method is so far only implemented in \emph{HELM-Flow} \footnote{http://www.gridquant.com/solutions/helm-flow/} by Gridquant. The in europe common tool \emph{PSS SINCAL} \footnote{http://www.simtec.cc/sites/sincal.asp} has not yet implemented this new approach, but it is the tool of choice for instance at the Institute of Power Transmission Systems at the Technische Universität München. Therefore, I implemented a tool which can apply \emph{HELM} to a power net stored in the file format of \emph{PSS SINCAL}.

The main results of this thesis are:
\begin{itemize}
	\item \emph{HELM} has a better convergence behaviour than the iterative methods.
	\item The theoretical perfect behaviour of \emph{HELM} can only be reached through a trade-off with respect to runtime.
\end{itemize}