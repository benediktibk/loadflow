\documentclass[12pt,a4paper]{article}

\usepackage[utf8]{inputenc}
\usepackage{amsmath}
\usepackage{amsfonts}
\usepackage{amssymb}
\usepackage{lmodern}
\usepackage{setspace}
\usepackage{siunitx}
\usepackage[ngerman]{babel}

\newcommand{\reffig}[1]{{Abb.~\ref{#1}}}
\newcommand{\refeq}[1]{{(\ref{#1})}}
 
\author{Benedikt Schmidt}
\title{Validierung von HELM}

\begin{document}
	\maketitle
	
	\section{Einführung}
	Klassische Verfahren zur Berechnung von Lastflüssen in elektrischen Energieverteilungsnetzen, wie zum Beispiel die Stromiteration oder Fast-Decoupled-Loadflow, weisen erhebliche Probleme in Bezug auf die Konvergenz auf. Zum einen können diese iterative Verfahren unter Umständen gar nicht konvergieren, obwohl das zu berechnende System an sich stabil wäre. Zusätzlich dazu können die berechneten Lösungen in bestimmten Situationen keine physikalische Lösung des Systems beschreiben. Holomorphic Embedding Loadflow (HELM) \cite{helmIEEE}, ein von Antonio Trias neu entwickelter Ansatz zur Lastflussberechnung, verspricht diese Probleme zu lösen. Die praktische Valdidierung dieser Aussagen erfolgte von mir durch die Entwicklung eines Programmes, welches sowohl HELM als auch die klassischen iterativen Verfahren implementiert.
	
	\section{Berechnung der Knotenspannungen in HELM}
	Ausgangsproblem ist ein elektrisches Energieverteilungsnetz bestehend aus Knoten bzw. sogenannten Bussen und Admittanzen $Y_{ij}$, welche die Knoten verbinden. Zudem muss an mindestens einem Knoten, dem Slackbus, Real- und Imaginärteil der Spannung zur Definition des Winkels vorgegeben sein. An allen anderen Knoten müssen dann jeweils genau zwei der folgenden Größen vorgegeben sein: $P_i$, $Q_i$, $\Re\{U_i\}$, $\Im\{U_i\}$ und $|U_i|$. Bei einer Vorgabe von $|U_i|$ und $P_i$ spricht man dann von einem PV-Bus, bei $P_i$ und $Q_i$ von einem PQ-Bus. Im folgenden werde ich mich immer auf den Fall eines PQ-Busses beziehen, die Modellierung eines PV-Busses in HELM wird in \cite{helmPV} behandelt. Die im weiteren Verlauf angeführten Formeln und Methoden entstammen größtenteils \cite{helmIEEE}, \cite{helmPatentApr2009} und \cite{helmPatentSept2009}.
	
	Für einen PQ-Bus ergibt sich dann aus der Kirchhoffschen Knotenregel
	\begin{equation}
		\sum_j Y_{ij} U_{j} = I_j + \frac{S_j^\star}{U_j^\star},
		\label{eq:loadflow}
	\end{equation}
	wobei die konstanten Ströme $I_j$ der Reduktion des Gesamtsystems um die Slackbusse entstammen.
	
	Die explizite Lösung von \refeq{eq:loadflow} ist an dieser Stelle aufgrund der quadratischen Form, gemischt mit der Konjugation der Knotenspannungen, nicht möglich. Aus diesem Grund heraus wird in \cite{helmPatentSept2009} eine Betrachtung der Spannungen als komplexe Funktionen in einem neuen Parameter $s$ vorgeschlagen. Um zudem zu erreichen, dass diese Funktionen holomporph sind, wird \refeq{eq:loadflow} zu 
	\begin{equation}
		\sum_j Y_{ij} U_{j}(s) = s I_j + \frac{s S_j^\star}{U_j^\star} + (1 - s) \sum_j Y_{ij}
		\label{eq:loadflowEmbedded}
	\end{equation}
	erweitert. Eine Auswertung dieser Gleichung an der Stelle $s = 1$ ergibt wiederum \refeq{eq:loadflow}, womit also über $U_i(s = 1)$ die eigentliche Knotenspannung $U_i$ berechnet werden kann. Das Ziel ist somit nicht mehr direkt die Berechnung von $U_i$, sondern anstelle dessen die Darstellung von $U_i(s)$. Um dies zu erreichen wird $U_i(s)$ durch die reduzierte Laurentreihe 
	\begin{equation}
		U_i(s) = \sum_{n = 0}^\infty c_{i,n} s^n
		\label{eq:powerSeries}
	\end{equation}
	in \refeq{eq:loadflowEmbedded} ersetzt. Damit lassen sich dann durch einen Koeffizientenvergleich Formeln für die sukzessive Berechnung von $c_{i,n}$ ermitteln.
	
	Die direkte Auswertung von \refeq{eq:powerSeries} ist an diesem Punkt meist nicht möglich, da die Reihe an der Stelle 0 entwickelt wurde und der Konvergenzradius in den meisten Fällen erheblich kleiner als 1 ist. Dieses Problem lässt sich allerdings über eine Methode zur analytischen Fortsetzung lösen, zum Beispiel mithilfe des Epsilon Algorithmuses von Wynn \cite{epsilonWynn}.
	
	Theoretisch lässt sich nun für die Lösung nachweisen, dass sie genau dann konvergiert wenn das System stabil ist, also kein Spannungszusammenbruch eintritt. Somit erhält man also an und für sich genau jenes Ergebnis, welches man sich wünscht: Wenn im System die Spannung zusammenbricht konvergiert das Verfahren nicht, falls das System stabil ist konvergiert das Verfahren. Praktisch ist man allerdings nicht in der Lage zum einen beliebig viele Koeffizienten zu berechnen, und zum anderen stößt man an die Grenze der Rechengenauigkeit. Bereits für sehr einfache Systeme erreichen die Koeffizienten $c_{i,50}$ dermaßen große Werte, dass mit einem 64 Bit Gleitkommadatentyp die Rechenfehler größer als etwaige Verbesserungen in der Genauigkeit des Ergebnisses ist. Auf Ansätze um dieses Problem zu umgehen bzw. zu lösen werde ich im nächsten Abschnitt eingehen.
	
	\section{Implementierung}
	
	\subsection{Software Architektur}
	
	\subsection{Datenbankschema}
	
	\subsection{Bedienung}
	
	\section{Ergebnisse}
	
	\begin{thebibliography}{1}
		\bibitem{helmIEEE}
			A.~Trias, \emph{The Holomorphic Embedding Load Flow Method}, IEEE PES General Meeting, July 2012
		\bibitem{helmPV}
			M.~K.~Subramanian, Y.~Feng, D.~Tylavsky, \emph{PV Bus Modeling in a Holomorphically Embedded Power-Flow Formulation}, 978-1-4799-1255-1/13, IEEE, 2013
		\bibitem{helmPatentApr2009} 
			A.~Trias, \emph{System and Method for Monitoring and Managing Electrical Power Transmission and Distribution Networks}, US Patent 7,519,506 B2, April 2009
		\bibitem{helmPatentSept2009} 
			A.~Trias, \emph{System and Method for Monitoring and Managing Electrical Power Transmission and Distribution Networks}, US Patent US 2009/0228154 A1, September 2009
			
		\bibitem{epsilonWynn}
			P.~Wynn, \emph{The Epsilon Algorithm and Operational Formulas of Numerical Analysis}, June 1960

	\end{thebibliography}

\end{document}